\chapter{Reinforcement learning}

\section{Introduction}

	\subsection{Setting}
	The learner is provided a set of possible states $\mathcal{S}$ and, for each state, a set of possible actions, $\mathcal{A}$ moving it to a new state.
	In performing action $a$ from state $s$, the learner is provided an immediate reward $r(s,a)$.
	The task is to learn a policy allowing to choose for each state $s$ the action $a$ maximizing the overall reward including future moves.
	The learner has to deal with problems of delayed rewards coming from future moves and trade-off between exploitation and exploration.
	Typical application include moving policies for robots and sequential scheduling problems in general.

	\subsection{Sequential decision making}
	An agent needs to take a sequence of decisions and should maximise some utility function.
	There is uncertainty in the result of a decision.

\section{Markov decision process MPD}
A Markov decision process is defined by:

\begin{itemize}
	\item A set of states $\mathcal{S}$ in which the agent can be at each time instant,
	\item A possibly empty set of terminal states $\mathcal{S}_G\subset\mathcal{S}$,
	\item A set of actions $\mathcal{A}$ that the agent can make,
	\item A transition model providing the probability of going to a state $s'$ with action $a$ from state $s$:

		$$p(s'|s,a)\qquad s,s'\in\mathcal{S},a\in\mathcal{A}$$

	\item A reward $R(s,a,s')$ for making action $a$ in state $s$ and reaching state $s'$
\end{itemize}

	\subsection{Utilities over time}
	An environment history is defined as a sequence of states.
	Utilities are defined over environment histories.
	An infinite horizon with no constraint on the number of time steps is assumed.
	A stationary preference where if one history is preferred to another at time $t$, the same should hold at time $t'$ provided they start from the same date is assumed.
	With these assumptions two sensible way to define utilities are:

	\begin{itemize}
		\item Additive rewards:

			$$U([s_0,\dots,s_n]) = R(s_0) + \cdots +R(s_n)$$

		\item Discounted rewards: for $\gamma\in[0,1]$:

			$$U([s_0,\dots,s_n]) = \gamma^0R(s_0) + \cdots +\gamma^{n-1}R(s_n)$$

	\end{itemize}

	Only rewards that only depend on the destination state are considered, in a more general case: $R(s_t, a_t, s_{t+1})$

	\subsection{Taking decisions}

		\subsubsection{Optimal policy}
		A policy $\pi$ is a full specification of what action to take at each state.
		The expected utility of a policy is the utility of an environment history taken in expectation over all possible histories generated with that policy.
		An optimal policy $\pi^*$ is a policy maximizing expected utility.
		For infinite horizons optimal policies are stationary and depends only on the current state.

			\paragraph{Utility of state}
			The utility of a state given a policy $\pi$ is:

			$$U^\pi(s) = \mathbb{E}_\pi\biggl[\sum\limits_{k=0}^\infty\gamma^tR(S_{t+k+1})|S_t=s\biggr]$$

			Where $S_{i+k+1}$ is the state reached after $k$ steps using policy $\pi$ starting from $S_t = s$.
			The true utility of a state is its utility under an optimal policy:

			$$U(s) = U^{\pi^*}(s)$$

			Given the true utility, an optimal policy is:

			$$\pi^*(s) = \arg\max\limits_{a\in\mathcal{A}}\sum\limits_{s'\in\mathcal{S}}p(s'|s,a)U(s')$$


	\subsection{Computing an optimal policy}
	The utility of a state is its immediate reward plus the expected discounted utility of the next state, assuming that the agent chooses an optimal action.

		\subsubsection{Bellman equation}

		$$U(s) = R(s) + \gamma\max\limits_{a\in\mathcal{A}}\sum\limits_{s'\in\mathcal{S}}p(s'|s,a)U(s')$$

		There is a Bellman equation for each state $s\in\mathcal{S}$ where utilities of states are solutions of the set of Bellman equations.
		The solutions to the set of Bellman equations are unique.
		Directly solving the set of equation is hard and because of the non-linearities introduced by the $\max$ operation.

		\subsubsection{Value iteration}

		\begin{enumerate}
			\item Initialize $U_0(s)$ to zero for all $s$.
			\item Repeat:

				\begin{enumerate}
					\item Do a Bellman update for each state $s$:

						$$U_{i+1}(s) \leftarrow R(s) + \gamma\max\limits_{a\in\mathcal{A}}\sum\limits_{s'\in\mathcal{S}}p(s'|s,a)U_i(s')$$

					\item $i += 1$
				\end{enumerate}

			\item Until max utility difference is below a threshold.
			\item Return $U$.
		\end{enumerate}

		The optimal policy can be set as:

		$$\pi^*(s) = \arg\max\limits_{a\in\mathcal{A}}\sum\limits_{s'\in\mathcal{S}}p(s'|s,a)U(s')$$

		\subsubsection{Policy iteration}

		\begin{enumerate}
			\item Initialize $\pi_0$ randomly.
			\item Repeat:

				\begin{enumerate}
					\item Policy evaluation, solve the set of linear equations:

						$$U_i(s) = R(s) + \gamma\sum\limits_{s'\in\mathcal{S}}p(s'|s,\pi_i(s))U_i(s')\qquad\forall s\in\mathcal{S}$$

					\item Policy improvement:

						$$\pi_{i+1}(s)\leftarrow \arg\max\limits_{a\in\mathcal{A}}\sum\limits_{s'\in\mathcal{S}}p(s'|s,a)U_i(s')\qquad\forall s\in\mathcal{S}$$

				\end{enumerate}

			\item Until there is no policy improvement.
			\item Return $\pi$
		\end{enumerate}

\section{Dealing with partial knowledge}
Value and policy iteration assume perfect knowledge of environment, transition model and rewards.
In most cases some of these aspects are not known.
Reinforcement learning aims at learning policies by space exploration.
With policy evaluation a policy is given and the environment is learned, with policy improvement both policy and environment are learned.

	\subsection{Policy evaluation in an unknown environment}

		\subsubsection{Adaptive dynamic programming}

			\paragraph{Algorithm}

			\begin{enumerate}
				\item Loop.

					\begin{enumerate}
						\item Initialize $s$.

							\begin{enumerate}
								\item Receive reward $r$, set $R(s) = r$.
								\item Choose next action $a\leftarrow\pi(s)$.
								\item Take action $a$, reach step $s'$.
								\item Update counts:

									$$N_{sa}\leftarrow N_{sa} + 1\qquad N_{s'|sa}\leftarrow N_{s'|sa}+1$$

								\item Update transition model:

									$$p(s''|s,a)\leftarrow \frac{N_{s''|sa}}{N_{sa}}\qquad\forall s''\in\mathcal{S}$$

								\item Update utility estimate:

									$$U\leftarrow PolicyEvaluation(\pi, U, p, R, \gamma)$$

							\end{enumerate}

						\item Until $s$ is terminal.
					\end{enumerate}
			\end{enumerate}

			\paragraph{Characteristics}
			The algorithm performs maximum likelihood estimation of transition probabilities and upon updating the transition model it calls a standard policy evaluation to update the utility estimate ($U$ is initially empty).
			Each step is expensive as it runs policy evaluation.

		\subsubsection{Temporal-difference policy evaluation}

			\paragraph{Rationale}
			Temporal-difference (TD) policy evaluation avoids running the policy evaluation at each iteration locally updating the utility.
			If transition from $s$ to $s'$ is observed if $s'$ was always the successor of $s$, its utility should be:

			$$U(s) = R(s) + \gamma U(s')$$

			The temporal-difference update rule updates the utility to get closer to that situation:

			$$U(s)\leftarrow U(s) + \alpha(R(s)+\gamma U(s')-U(s))$$

			Where $\alpha$ is a learning rate, possibly decreasing over time.

			\paragraph{Algorithm}

			\begin{enumerate}
				\item Loop.

					\begin{enumerate}
						\item Initialize $s$
						\item Repeat.

							\begin{enumerate}
								\item Receive reward $r$.
								\item Choose next action $a\rightarrow \pi(s)$ and reach step $s'$.
								\item Update the local utility estimate:

									$$U(s)\leftarrow U(s) + \alpha(r+\gamma U(s')-U(s))$$

							\end{enumerate}

						\item Until $s$ is terminal
					\end{enumerate}

			\end{enumerate}

			\paragraph{Characteristics}
			There is no need for a transition model for the utility update and each step is faster than ADP, without loosing anything on the long run.
			It takes longer to converge and can be seen as a rough efficient approximation of ADP.

	\subsection{Policy learning in an unknown environment}

		\subsubsection{Setting}
		Policy learning requires to combine the learning of the environment and the learning of the optimal policy for it.
		A simple option consist of replacing the policy evaluation in ADP with optimal policy computation given the current knowledge of the environment creating a greedy agent:

		$$U(s) = R(s) + \gamma\max\limits_{a\in\mathcal{A}}\sum\limits_{s'\in\mathcal{S}}p(s'|s,a)U(s')$$

		Because the knowledge of the environment is incomplete a greedy agent usually learns a suboptimal policy because of lack of exploration.

		\subsubsection{Exploration-exploitation trade-off}
		Exploitation consists in following promising directions given the current knowledge and exploration consists in trying novel directions looking for better unknown alternatives.
		A reasonable trade-off should be used in defining the search scheme.
		For example an $\epsilon$-greedy strategy chooses a random move with probability $\epsilon$ and is greedy otherwise.
		Moreover a higher utility estimate can be assigned to relatively unexplored state-action pairs:

		$$U^+(s) = R(s) + \gamma\max\limits_{a\in\mathcal{A}}f\biggl(\sum\limits_{s'\in\mathcal{S}}p(s'|s,a)U^+(s'),N_{sa}\biggr)$$

		With $f$ increasing over the first argument and decreasing over the second.

		\subsubsection{Temporal-difference learning - learning utilities of actions}
		The TD policy evaluation can also be adapted to learn an optimal policy.
		If TD is used to learn a state utility function it needs to estimate a transition model to derive a policy.
		TD can instead be applied to learn an action utility function $Q(s,a)$ and the optimal policy corresponds to:

		$$\pi^*(s) = \arg\max\limits_{a\in\mathcal{A}}Q(s,a)$$

		\subsubsection{SARSA - on-policy temporal-difference learning}

		\begin{enumerate}
			\item Loop.

				\begin{enumerate}
					\item Initialize $s$.
					\item Repeat.

						\begin{enumerate}
							\item Receive reward $r$.
							\item Choose next action $a\leftarrow\pi^\epsilon(s)$.
							\item Take action $a$ and reach step $s'$.
							\item Choose action $a'\leftarrow\pi^\epsilon(s')$.
							\item Update local utility estimate:

								$$Q(s,a) \leftarrow Q(s,a) + \alpha(r+\gamma Q(s',a')- Q(s,a))$$

						\end{enumerate}

					\item Until $s$ is terminal.
				\end{enumerate}

		\end{enumerate}

		$\pi^\epsilon$ is an $\epsilon$-greedy or some other form of non-greedy policy based on $Q$.

		\subsubsection{Q-learning - off-policy temporal-difference learning}

		\begin{enumerate}
			\item Loop.

				\begin{enumerate}
					\item Initialize $s$.
					\item Repeat.

						\begin{enumerate}
							\item Receive reward $r$.
							\item Choose next action $a\leftarrow\pi^\epsilon(s)$.
							\item Take action $a$ and reach step $s'$.
							\item Update local utility estimate:

								$$Q(s,a) \leftarrow Q(s,a) + \alpha(r+\gamma Q(s',a')- Q(s,a))$$

						\end{enumerate}

					\item Until $s$ is terminal.
				\end{enumerate}

		\end{enumerate}

		\subsubsection{SARSA vs Q-learning}
		SARSA is on-policy: it updates $Q$ using the current policy's action.
		Q-learning is off-policy: it updates $Q$ using the greedy policy's action.
		Off-policy methods are more flexible and they can learn from traces generated with an unknown policy.
		On-policy methods tend to converge faster and are easier to use for continuous-state spaces and linear function approximators.

\section{Scaling to large state spaces}

	\subsection{Function approximation}
	All techniques seen so far assume a tabular representation of utility functions.
	This type of representation doesn't scale to large state spaces.
	The solution is to rely on function approximation: $U(s)$ or $Q(s,a)$ are approximated with a parametrized function.
	The function takes a state representation as input and allows to generalize to unseen states.

	\subsection{Learning the approximation function}

		\subsubsection{Temporal-difference learning - state utility}
		The TD error is:

		$$E(s,s') = \frac{1}{2}(R(s) + \gamma U_{\vec{\theta}}(s')-U_{\vec{\theta}}(s))^2$$

		The error gradient with respect to the function's parameter is:

		$$\nabla_{\vec{\theta}} E(s,s') = (R(s) + \gamma U_{\vec{\theta}}(s') - U_{\vec{\theta}}(s))(-\nabla_{\vec{\theta}}U_{\vec{\theta}}(s))$$

		So the stochastic gradient update rule is:

		\begin{align*}
			\vec{\theta} &= \vec{\theta} - \alpha\nabla_{\vec{\theta}}E(s,s') = \\
				     &= \vec{\theta} + \alpha(R(s) + \gamma U_{\vec{\theta}}(s') - U_{\vec{\theta}}(s))(\nabla_{\vec{\theta}}U_{\vec{\theta}}(s))
		\end{align*}

		\subsubsection{Temporal-difference learning - action utility Q-learning}
		The TD error is:

		$$E((s,a),s') = \frac{1}{2}(R(s) + \gamma\max\limits_{a'\in\mathcal{A}}Q_{\vec{\theta}}(s',a')-Q_{\vec{\theta}}(s,a))^2$$

		The error gradient with respect to the function's parameter is:

		$$\nabla_{\vec{\theta}} E((s,a),s') = (R(s) + \gamma\max\limits_{a'\in\mathcal{A}}Q_{\vec{\theta}}(s',a') - Q_{\vec{\theta}}(s,a))(-\nabla_{\vec{\theta}}Q_{\vec{\theta}}(s,a))$$

		So the stochastic gradient update rule is:

		\begin{align*}
			\vec{\theta} &= \vec{\theta} - \alpha\nabla_{\vec{\theta}}E((s,a),s') = \\
				     &= \vec{\theta} + \alpha(R(s) + \gamma\max\limits_{a'\in\mathcal{A}}Q_{\vec{\theta}}(s',a') - Q_{\vec{\theta}}(s,a))(\nabla_{\vec{\theta}}Q_{\vec{\theta}}(s,a))
		\end{align*}
