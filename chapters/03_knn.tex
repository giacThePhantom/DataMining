\chapter{K-nearest neighbours}

\section{Introduction}
The $K$-nearest neighbours is an algorithm that, given a training set represented as a vector of features, gives the label for a new sample as label of the majority of the $K$ nearest sample in the training set.

\section{Measuring the instance between instances}

	\subsection{Metric or distance definition}
	Given a set $\mathcal{X}$ a function $d:\mathcal{X}\times\mathcal{X}\rightarrow\mathbb{R}^+_0$ is a metric for $\mathcal{X}$ if for any $x,y,z\in \mathcal{X}$ the following properties are satisfied:
	\begin{itemize}
		\item Reflexivity $d(x,y) = 0 \Leftrightarrow x = y$.
		\item Symmetry $d(x,y) = d(y,x)$.
		\item Triangle inequality $d(x, y) + d(y, z) \ge d(x, z)$.
	\end{itemize}

	\subsection{Euclidean distance}
	The euclidean distance in $\mathbb{R}^n$ is:
	$$d(x,y) = \sqrt{\sum\limits_{i=1}^n(x_i-y_i)^2}$$

\section{Algorithms}

	\subsection{Classification}

	\subsection{Regression}

\section{Characteristics}

	\begin{multicols}{2}
		\begin{itemize}
			\item Instance-based learning: the model used for prediction is calibrated for the test example to be processed.
			\item Lazy learning: the computation is mostly deferred to the classification phase.
			\item Local learner: assumes prediction should mainly influenced by nearby instances.
			\item Uniform feature weighting: all feature are uniformly weighted in computing dinstances.
		\end{itemize}
	\end{multicols}
