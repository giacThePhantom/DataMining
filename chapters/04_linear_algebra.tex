\chapter{Linear algebra}

\section{Vector space}
A set $\mathcal{X}$ is called a vector space over $\mathbb{R}$ if addition and scalar multiplication are defined and satisfy for all $x,y,z\in\mathcal{X}$ and $\lambda,\mu\in\mathcal{R}$:

\begin{multicols}{2}
	\begin{itemize}
		\item Addition:
			\begin{itemize}
				\item Association: $x+(y+z) = (x+y)+z$.
				\item Commutation: $x+y = y+x$.
				\item There is an identity element: $\exists 0\in\mathcal{X}:x+0 = x$.
				\item There is an inverse element: $\forall x\in\mathcal{X} \exists x'\in\mathcal{X}:x+x'=0$.
			\end{itemize}
		\item Scalar multiplication:
			\begin{itemize}
				\item Is distributive over elements: $\lambda(x+y) = \lambda x + \lambda y$.
				\item Is distributive over scalars: $(\lambda+\mu)x = \lambda x + \mu x$.
				\item Is associative over scalars: $\lambda(\mu x) = (\lambda\mu)x$.
				\item There is an identity element: $\exists 1\in\mathbb{R}: 1x=x$.
			\end{itemize}
	\end{itemize}
\end{multicols}

	\subsection{Properties and operations}

		\subsubsection{Subspace}
		A subspace is any non-empty subset of $\mathcal{X}$ being itself a vector space.

		\subsubsection{Linear combination}
		Given $\lambda_i\in\mathbb{R}\land x_i\in\mathcal{X}$, a linear combination is:
		$$\sum\limits_{i=1}^n\lambda_i x_i$$

		\subsubsection{Span}
		The span of vectors $x_1, \dots, x_n$ is defined as the set of their linear combination:
		$$\bigl\{\sum\limits_{i=1}^n\lambda_ix_i, \lambda_i\in\mathbb{R}\bigr\}$$

		\subsubsection{Linear independence}
		A set of vector $x_i$ is linearly independent if none of them can be written as a linear combination of the others.

	\subsection{Basis}
	A set of vectors $x_i$ is a basis for $\mathcal{X}$ if any element in $\mathcal{X}$ can be uniquely written as a linear combination of vectors $x_j$.
	The vectors $x_j$ need to be linearly independent.
	All bases of $\mathcal{X}$ have the same number of elements, called the dimension of the vector space.

\section{Matrices}

	\subsection{Linear maps}
	Given two vector spaces $\mathcal{X}$ and $\mathcal{Z}$ a function $f:\mathcal{X}\rightarrow\mathcal{Z}$ is a linear map if $\forall x, y\in\mathcal{X}\lambda\in\mathbb{R}$:
	\begin{multicols}{2}
		\begin{itemize}
			\item $f(x + y) = f(x) + f(y)$.
			\item $f(\lambda x) = \lambda f(x)$.
		\end{itemize}
	\end{multicols}

	\subsection{Linear maps as matrices}
	A linear map between two finite dimensional spaces $\mathcal{X}$ and $\mathcal{Z}$ of dimension $n$ and $m$ can always be written as a matrix.
	Let $\{x_1, \dots, x_n\}$ and $\{z_1, \dots, z_m\}$ be some bases for $\mathcal{X}$ and $\mathcal{Z}$ respectively.
	For any $x\in\mathcal{X}$:
	\begin{align*}
		f(x) &= f(\sum\limits_{i=1}^n\lambda_ix_i) = \sum\limits_{i = 1}^n\lambda_if(x_i)\\
		f(x_i) &= \sum\limits_{j=1}^ma_{ji}^ma_{ij}z_j\\
		f(x) &= \sum\limits_{i=1}^n\sum\limits_{j=1}^m\lambda_ia_{ji}z_j = \sum\limits_{j=1}^m\bigl(\sum\limits_{i=1}^n\lambda_ia_{ji}\bigr)z_j = \sum\limits_{j=1}^m\mu_j z_j
	\end{align*}

		\subsubsection{Matrix of basis transformation}
		A matrix can be used to transform the basis is:

		$$M\in\mathbb{R}^{m\times n} = \begin{bmatrix} a_{11} & \cdots & a_{1n} \\ \vdots & \vdots & \vdots \\ a_{m1} & \cdots & a_{mn} \end{bmatrix}$$

		The mapping from basis to basis coefficient is done:

		$$M\lambda = \mu$$

		\subsubsection{Matrix changing the coordinates, 2D examples}
		Let $B = \biggl\{ \begin{bmatrix} 1 \\0 \end{bmatrix} , \begin{bmatrix} 0 \\ 1\end{bmatrix}\biggr\}$ be the standard basis in $\mathbb{R}^2$ and $B' = \biggl\{\begin{bmatrix} 3 \\ 1 \end{bmatrix}, \begin{bmatrix} -2 \\ 1 \end{bmatrix}\biggr\}$ be an alternative basis.
		The change of coordinate matrix from $B'$ to $B$ is: 

		$$P = \begin{bmatrix} 3 & -2 \\ 1 & 1 \end{bmatrix}$$

		So that: 
		
		$$[v]_B = P[v]_{B'}\qquad\land\qquad [v]_{B'} = P^{-1}[v]_B$$

		For arbitrary $B$ and $B'$ $P$'s columns must be the $B'$ vectors written in terms of the $B$ ones.

	\subsection{Matrix properties}

		\subsubsection{Transpose}
		The transpose matrix is the matrix obtained exchanging the rows with column $M^T$.

		$$(MN)^T = N^TM^T$$

		\subsubsection{Trace}
		The trace is the sum of the diagonal elements of a matrix:

		$$tr(M) = \sum\limits_{i = 1}^n M_{ii}$$

		\subsubsection{Inverse}
		The inverse is the matrix which multiplied with the original matrix gives the identity:

		$$MM^{-1} = I$$

		\subsubsection{Rank}
		The rank of an $n\times m$ matrix is the dimension of the space spanned by its columns.

	\subsection{Matrix derivatives}

	\begin{align*}
		\frac{\partial M_X}{\partial x} &= M\\
		\frac{\partial y^T Mx}{\partial x} &=M^Ty\\
		\frac{\partial x^T Mx}{\partial x} &= (M^T + M)x\\
		\frac{\partial x^T Mx}{\partial x} &= 2Mx \qquad \text{if }M\text{ is symmetric}\\
		\frac{\partial x^T x}{\partial x} &= 2x
	\end{align*}

	Results are columns vectors.
	Transposing the matrix gives the row vectors.

	\subsection{Metric structure}

		\subsubsection{Norm}
		A function $||\cdot ||: \mathcal{X} \rightarrow \mathbb{R}^+_0$ is a norm $\forall x,y\in\mathcal{X},\lambda\in\mathbb{R}$:

		\begin{multicols}{3}
			\begin{itemize}
				\item $||x + y|| \le ||x|| + ||y||$
				\item $||\lambda x || = |\lambda| ||x||$
				\item $||x||>0 if x\neq 0$
			\end{itemize}
		\end{multicols}

		\subsubsection{Metric}
		A norm defines a metric $d: \mathcal{X}\times\mathcal{X}\rightarrow\mathbb{R}_0^+$:

		$$d(x,y) = ||x - y||$$

		Not every metric gives rise to a norm.

	\subsection{Dot product}
	A dot product $\langle \cdot, \cdot\rangle:\mathcal{X}\times\mathcal{X}\rightarrow\mathbb{R}$ is a symmetric bilinear form which is positive semi-definite:

	$$\langle x, x\rangle \ge 0\forall x \in\mathcal{X}$$

	A positive definite dot product satisfies:

	$$\langle x, x\rangle = 0 \Leftrightarrow x = 0$$

		\subsubsection{Norm}
		Any dot product defines a corresponding norm via:

		$$||x|| = \sqrt{\langle x, x\rangle}$$

		\subsubsection{Properties}

			\paragraph{Angle}
			The angle $\theta$ between two vectors is defined as:

			$$\cos\theta = \frac{\langle x, z\rangle}{||x||||z||}$$

			\paragraph{Orthogonal}
			Two vectors are orthogonal if $\langle x, y\rangle = 0$.

			\paragraph{Orthonormal}
			A set of vectors $\{x_1, \dots, x_n\}$ is orthonormal is

			$$\langle x_i, x_j\rangle = \delta_{ij}$$

			Where $\delta_{ij} = 1$ if $i = j$, $0$ otherwise.
			If $x$ and $y$ are $n$-dimensional column vectors, their dot product is computed as:

			$$\langle x, y\rangle = x^Ty = \sum\limits_{i=1}^nx_iy_i$$

\section{Eigenvalues and eigenvectors}
Given a $n\times n$ matrix $M$ the real value $\lambda$ and non zero vector $x$ are eigenvalue and the corresponding eigenvector of $M$ if:

$$Mx = \lambda x$$

	\subsection{Cardinality}
	An $n\times n$ matrix has $n$ eigenvalues, but less than $n$ distinct ones.
	The number of eigenvalues is the number of linear independent eigenvectors.

	\subsection{Singular matrices}
	A matrix is singular if it has a zero eigenvalue:

	$$Mx = 0x = 0$$

	A singular matrix has linearly dependent columns.

	\subsection{Symmetric matrices}
	Eigenvectors corresponding to distinct eigenvalues are orthogonal:

	\begin{align*}
		\lambda\langle x, z\rangle &= \langle Ax, z\rangle=\\
					   &=(Ax)^Tz=
					   &=x^TA^Tz=\\
					   &=x^TAz=\\
					   &\langle x, Az\rangle=\\
					   &=\mu\langle x, z\rangle
	\end{align*}

	\subsection{Eigen-decomposition}

		\subsubsection{Raleigh quotient}

		\begin{align*}
			Ax &=\lambda x\\
			\frac{x^TAx}{x^Tx} &=\lambda\frac{x^Tx}{x^Tx} = \lambda
		\end{align*}

		\subsubsection{Finding eigenvector}
		To find the eigenvector you need to maximize the eigenvalue:

		$$x = \max\limits_v\frac{v^TAv}{v^Tv}$$

		After that you need to normalize it so the solution is invariant to rescaling:

		$$x\leftarrow\frac{x}{||x||}$$

		\subsubsection{Deflating matrix}

		$$\bar{A} = A - \lambda xx^T$$

		The deflation turns $x$ into a zero eigenvalue eigenvector:

		\begin{align*}
			\bar{A}x &= Ax -\lambda xx^Tx\\
				 &Ax-\lambda x = 0
		\end{align*}
		
		Other eigenvalues are unchanged as eigenvectors with distinct eigenvalues are orthogonal because the matrix is symmetric:

		\begin{align*}
			\bar{A}z &= Az-\lambda xx^Tz\\
			\bar{A}z &=Az
		\end{align*}

		\subsubsection{Iterating}
		The maximization procedure is repeated on the deflated matrix until solution is zero.
		Minimization is iterated to get eigenvectors with negative eigenvalues.
		Eigenvectors with zero eigenvalues are obtained extending the obtained set to an orthonormal basis.
		
		\subsubsection{Eigen-decomposition}
		Let $V = [v_1 \dots v_n]$ be a matrix with orthonormal eigenvectors as columns.
		Let $A$ be the diagonal matrix of corresponding eigenvalues.
		A square symmetric matrix can be diagonalizes as:

		$$V^TAV=\Lambda$$

		A diagonalized matrix is simpler to manage and has the same properties as the original one.

			\paragraph{Proof}

			\begin{align*}
				A[v_1\dots v_n] &= [v_1\dots v_n \begin{bmatrix}\lambda_1 & & 0\\ & \ddots & \\ 0 & & \lambda_n\end{bmatrix}\\
				AV &= V\Lambda\\
				V^{-1}AV &= V^{-1}V\Lambda\\
				V^TAV &= \Lambda
			\end{align*}

			$V$ is a unitary matrix with orthonormal columns for which $V^{-1}=V^T$.

		\subsubsection{Positive semi-definite matrix}
		An $n\times n$ symmetric matrix $M$ is positive semi-definite if all its eigenvalues are non-negative.
		Positive semi-definite:
		\begin{multicols}{2}
			\begin{itemize}
				\item $\Leftrightarrow\forall x\in\mathbb{R}^n: x^TMx\ge 0$
				\item $\Leftrightarrow\exists B: M=B^TB$
			\end{itemize}
		\end{multicols}

		\subsubsection{Scaling transformation in standard basis}
		Let $x_i = [1,0]$ and $x_2 = [0,1]$ the standard orthonormal basis in $\mathbb{R}^2$.
		Let $x = [x_1, x_2]$ am arbitrary vector in $\mathbb{R}^2$.
		A linear transformation is a scaling transformation if it only stretches $x$ along its directions.

		\subsubsection{Scaling transformation in eigenbasis}
		Let $A$ be a non-scaling transformation in $\mathbb{R}$.
		Let $\{v_1, v_2\}$ be an eigenbasis for $A$.
		By representing vectors in $\mathbb{R}^2$ in terms of the $\{v_1, v_2\}$ basis $A$ becomes a scaling transformation.

\section{Principal component analysis}
Principal component analysis or PCA is a non-supervised machine learning technique that accomplishes dimensionality reduction.
Let $X$ be a data matrix with correlated coordinates.
PCA is a linear transformation mapping data to a system of uncorrelated coordinates.
It corresponds to fitting an ellipsoid to the data, whose axis are the coordinates of the new space.

	\subsection{Procedure}
	Given a dataset $X\in\mathbb{R}^{n\times d}$ in $d$ dimension:
	Compute the mean of the data:

	$$\bar{x} = \frac{1}{n}\sum\limits_{i=1}^nX_i$$

	Center the data into the origin:

	$$X - \begin{bmatrix}\bar{x}\\\cdots\\\bar{x}\end{bmatrix}$$

	Compute the data covariance:

	$$C = \frac{1}{n}X^TX$$

	Compute the orthonormal eigen-decomposition of $C$:

	$$V^TCV=A$$

	Us it as the new coordinate system:

	$$x' = V^{-1}x = V^Tx$$

	This method assumes linear correlation and Gaussian distribution.

	\subsection{Dimensionality reduction}
	Each eigenvalue corresponds to the amount of variance in that direction.
	Select only the $k$ eigenvalues with largest eigenvalue for dimensionality reduction:
	$$W = [v_1, \dots, v_k]$$
	$$x' = W^Tx$$
