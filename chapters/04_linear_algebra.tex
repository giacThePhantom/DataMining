\chapter{Linear algebra}

\section{Vector space}
A set $\mathcal{X}$ is called a vector space over $\mathbb{R}$ if addition and scalar multiplication are defined and satisfy for all $x,y,z\in\mathcal{X}$ and $\lambda,\mu\in\mathcal{R}$:

\begin{multicols}{2}
	\begin{itemize}
		\item Addition:
			\begin{itemize}
				\item Association: $x+(y+z) = (x+y)+z$.
				\item Commutation: $x+y = y+x$.
				\item There is an identity element: $\exists 0\in\mathcal{X}:x+0 = x$.
				\item There is an inverse element: $\forall x\in\mathcal{X} \exists x'\in\mathcal{X}:x+x'=0$.
			\end{itemize}
		\item Scalar multiplication:
			\begin{itemize}
				\item Is distributive over elements: $\lambda(x+y) = \lambda x + \lambda y$.
				\item Is distributive over scalars: $(\lambda+\mu)x = \lambda x + \mu x$.
				\item Is associative over scalars: $\lambda(\mu x) = (\lambda\mu)x$.
				\item There is an identity element: $\exists 1\in\mathbb{R}: 1x=x$.
			\end{itemize}
	\end{itemize}
\end{multicols}

	\subsection{Properties and operations}

		\subsubsection{Subspace}
		A subspace is any non-empty subset of $\mathcal{X}$ being itself a vector space.

		\subsubsection{Linear combination}
		Given $\lambda_i\in\mathbb{R}\land x_i\in\mathcal{X}$, a linear combination is:
		$$\sum\limits_{i=1}^n\lambda_i x_i$$

		\subsubsection{Span}
		The span of vectors $x_1, \dots, x_n$ is defined as the set of their linear combination:
		$$\bigl\{\sum\limits_{i=1}^n\lambda_ix_i, \lambda_i\in\mathbb{R}\bigr\}$$

		\subsubsection{Linear independence}
		A set of vector $x_i$ is linearly independent if none of them can be written as a linear combination of the others.

	\subsection{Basis}
	A set of vectors $x_i$ is a basis for $\mathcal{X}$ if any element in $\mathcal{X}$ can be uniquely written as a linear combination of vectors $x_j$.
	The vectors $x_j$ need to be linearly independent.
	All bases of $\mathcal{X}$ have the same number of elements, called the dimension of the vector space.

\section{Matrices}

	\subsection{Linear maps}
	Given two vector spaces $\mathcal{X}$ and $\mathcal{Z}$ a function $f:\mathcal{X}\rightarrow\mathcal{Z}$ is a linear map if $\forall x, y\in\mathcal{X}\lambda\in\mathbb{R}$:
	\begin{multicols}{2}
		\begin{itemize}
			\item $f(x + y) = f(x) + f(y)$.
			\item $f(\lambda x) = \lambda f(x)$.
		\end{itemize}
	\end{multicols}

	\subsection{Linear maps as matrices}
	A linear map between two finite dimensional spaces $\mathcal{X}$ and $\mathcal{Z}$ of dimension $n$ and $m$ can always be written as a matrix.
	Let $\{x_1, \dots, x_n\}$ and $\{z_1, \dots, z_m\}$ be some bases for $\mathcal{X}$ and $\mathcal{Z}$ respectively.
	For any $x\in\mathcal{X}$:
	\begin{align*}
		f(x) &= f(\sum\limits_{i=1}^n\lambda_ix_i) = \sum\limits_{i = 1}^n\lambda_if(x_i)\\
		f(x_i) &= \sum\limits_{j=1}^ma_{ji}^ma_{ij}z_j\\
		f(x) &= \sum\limits_{i=1}^n\sum\limits_{j=1}^m\lambda_ia_{ji}z_j = \sum\limits_{j=1}^m\bigl(\sum\limits_{i=1}^n\lambda_ia_{ji}\bigr)z_j = \sum\limits_{j=1}^m\mu_j z_j
	\end{align*}
